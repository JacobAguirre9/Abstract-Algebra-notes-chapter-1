\documentclass{article}
\usepackage[utf8]{inputenc}
\usepackage[margin=0.75in]{geometry}

\begin{document}
	\begin{center}
    
    	% MAKE SURE YOU TAKE OUT THE SQUARE BRACKETS
    
		\LARGE{\textbf{Abstract Algebra Chapter 1 Notes}} \\
        \vspace{1em}
        \Large{Introduction to Abstract Algebra} \\
        \vspace{1em}
        \normalsize\textbf{Jacob Aguirre} \\
        \normalsize{jaguirre31@gatech.edu} \\
        \vspace{1em}
        \normalsize{Advisor: Dr. Daniel Dench, School of Economics} \\
        \vspace{1em}
        \normalsize{Georgia Institute of Technology, Atlanta, GA} \\
        \normalsize{Bachelors of Science in Economics,Mathematics}
     
	\end{center}
    \begin{normalsize}
    
    	\section{Introduction:}
    	
    	\text\hspace{5mm}We begin with the question,"What is Algebra". Algebra is the beginner, and the end. It permeates through all our mathematical intuitions. Consider the set $\mathbb{N}:=\{1,2,...,N\}$ which is defined to be the natural numbers. This set $\mathbb{N}$ comes with the operations of addition and multiplication $(+,x)$. Now, consider the set $\mathbb{Z}:=\{...,-2,-1,0,1,2,...\}$, which is defined to be the integers.\\ 
    	
    	\text We can form these geometric intuitions by simply thinking of the Naturals as a line in 2d space. We can consider the set $\mathbb{Q}:=\{\frac{a}{b}|a,b\in\mathbb{Z},b\neq0\}$, which are the rational numbers. The rational numbers are also equipped with the operations of addition and multiplication $(+,x)$. This allows for us to define that non-zero elements contain multiplicative inverses.
       
       
        
        
\end{normalsize}
  
\end{document}
